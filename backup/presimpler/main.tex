\documentclass[12pt]{article}

\usepackage{amsmath, amsthm, amssymb}
\usepackage{amssymb}
\usepackage{bbm}

\usepackage[hidelinks]{hyperref}
\usepackage{cleveref}
\usepackage{url}

%%% For figures
\usepackage{graphicx}
\DeclareGraphicsExtensions{.pdf,.png,.jpg}

%%% For tables
\usepackage[table]{xcolor}

%specifying table and figure packages
\usepackage{caption}

\usepackage[T1]{fontenc}
\usepackage[all]{xy}
\usepackage[inline]{enumitem}

\usepackage{tikz-cd}

\newtheorem{theorem}{Theorem}
\newtheorem{corollary}{Corollary}
\newtheorem{remark}{Remark}
\newtheorem{example}{Example}
\newtheorem{definition}{Definition}
\newtheorem{proposition}{Proposition}
\newtheorem{lemma}{Lemma}

\newcommand{\mattia}[1]{\textcolor{cyan}{#1}}
\newcommand{\pietro}[1]{\textcolor{teal}{#1}}
\newcommand{\N}{{\mathbb{N}}}
\newcommand{\Ban}{{\mathbf{Ban}}}
\newcommand{\Top}{{\mathbf{Top}}}
\newcommand{\pt}{{\textnormal{pt}}}
\newcommand{\Hom}{{\textnormal{Hom}}}
\newcommand{\End}{{\textnormal{End}}}
\newcommand{\Fun}{{\textnormal{Fun}}}
\newcommand{\Aut}{{\textnormal{Aut}}}
\newcommand{\Obj}{{\textnormal{Obj}}}
\newcommand{\id}{{\textnormal{id}}}
\newcommand{\Morph}{{\textnormal{Morph}}}
\newcommand{\Set}{{\mathbf{Set}}}
\newcommand{\Cat}{{\mathbf{C}}}
\newcommand{\DCat}{{\mathbf{D}}}
\newcommand{\JCat}{{\mathcal{J}}}
\newcommand{\FCoalg}{{F\textnormal{-coalg}}}
\newcommand{\stCoalg}{{s_!t^*\textnormal{-coalg}}}
\newcommand{\range}[2]{{\{{#1}, \dots,{#2}\}}}
\newcommand{\anon}{{\,\mbox{-}\,}}

\crefname{diagram}{diag.}{diags.}
\Crefname{diagram}{Diagram}{Diagrams}
\creflabelformat{diagram}{#2(#1)#3}

\title{Weight sharing via comonads}
\author{
    Pietro Vertechi \and Mattia G. Bergomi
}
\date{}
\begin{document}
\maketitle
\begin{abstract}
\end{abstract}

\section{Introduction}

One of the key reasons for the success of neural networks is the notion of {\em weight sharing}. In its simplest form, one can consider, among maps $[M, X] \rightarrow [M, X]$, those that originate from maps $[M, X] \rightarrow X$. Indeed, given
\begin{equation*}
    {\cal F} \colon [M, X] \rightarrow X,
\end{equation*}
one can consider the map
\begin{align*}
    \widehat{\cal F} \colon [M, X]       & \rightarrow [M, X]              \\
    \left(\widehat{\cal F}\phi\right)(m) & = {\cal F}(a \mapsto \phi(ma)).
\end{align*}

This simple notion of weight sharing does not seem sufficient to cover complex neural networks, made up of several nodes, each with their own symmetries, which interact with each other in non-trivial ways.

We aim to show that the a similar strategy, based on a different comonad, is sufficient to recover, among other examples, classical recurrent and convolutional neural networks, incorporating at the same time equivariance and locality.

\section{Categorical preliminaries}

\begin{definition}\label{def:cartesian_category}
    We say that a category is {\em Cartesian} if it has finite products. We say that a functor between Cartesian categories is {\em Cartesian} if it preserves finite products.
\end{definition}

\pietro{TODO definisci anche locally Cartesian closed!}

\begin{proposition}\label{prop:internal_category_comonad}
    Let $\DCat$ be a locally Cartesian closed category. Then, there is a well defined notion of category {\em internal} to $\DCat$~\cite{mac2013categories}.
    Let $\JCat = s, t \colon C_1 \rightrightarrows C_0$ be such an internal category. $\JCat$ induces a limit-preserving comonad
    \begin{equation*}
        t_!s^*\colon \DCat/C_0 \rightrightarrows \DCat/C_0.
    \end{equation*}
    \pietro{TODO: definisci * e ! (base change functors).}
\end{proposition}

\begin{proof}
    The map $s_*t^*\colon \DCat/C_0 \rightrightarrows \DCat/C_0$ is a monad~\cite[Thm.~V.8.2]{Mac_Lane_1994}. As $\DCat$ is locally Cartesian closed, $s^*$ has a right adjoint $s_!$, hence we can consider $t_!s^*$. As $T = t_!s^*$ is the right adjoint of the monad $s_*t^*$, it is a comonad. Being a right adjoint, $T$ preserves limits.
\end{proof}

\begin{definition}\label{def:nilpotent_functor}
    A functor $F$ is {\em nilpotent} if there exists $n \in \N$ such that the composed functor $F^n$ maps all objects to the terminal object.
\end{definition}

\begin{definition}\label{def:internal_quiver_bounded_depth}
    Let $\DCat$ be a locally Cartesian closed category. An internal quiver
    \begin{equation*}
        s, t\colon E \rightrightarrows V
    \end{equation*}
    {\em has bounded depth} if there exists $n \in \N$ such that the iterated pullback
    \begin{equation}\label{eq:nilpotent_left_adjoint}
        \underbrace{E \times_V E \times_V \dots \times_V E}_{n \text{ factors}}
    \end{equation}
    is initial in $\DCat/V$.
\end{definition}

\begin{proposition}\label{prop:internal_quiver_functor}
    Let $\DCat$ be a locally Cartesian closed category. Let $s, t\colon E \rightrightarrows V$ be an quiver of bounded depth. Then
    \begin{equation*}
        s_!t^*\colon \DCat/V \rightarrow \DCat/V
    \end{equation*}
    is a limit-preserving nilpotent functor.
\end{proposition}

\begin{proof}
    The functor $s_!t^*$ is the right adjoint of the functor  $t_*s^*\colon \DCat/V \rightarrow \DCat/V$, hence it preserves limits and is therefore Cartesian. By~\cref{eq:nilpotent_left_adjoint}, $(t_*s^*)^n$ maps all objects to the initial object, so its adjoint $(s_!t^*)^n$ maps all objects to the terminal object.
\end{proof}

\section{Equivariant architectures}

Let us now assume that $\DCat$  is a locally Cartesian closed category with a terminal object (and thus all finite limits).

\begin{definition}\label{def:equivariance_category}
    Let $M$ be a monoid in $\DCat$ and $I \in \Obj$. The {\em equivariance category} $\JCat_{M, I}$ is given by the projection maps
    \begin{equation*}
        s, t\colon M \times I \times I \rightrightarrows I,
    \end{equation*}
    with composition and identity given by composition and identity in $M$. The {\em equivariance comonad} is the comonad induced by this category via~\cref{prop:internal_category_comonad}, that is, $t_!s^*$.
\end{definition}

We introduce the notion of {\em nilpotent constraint}, which we will need in order to limit equivariance and connectivity. To do so, we need some additional definitions.

\begin{definition}\label{def:nilpotent_architecture}
    Let $\JCat_{M, I}$ be an equivariance category. A {\em nilpotence constraint} on $\JCat_{M, I}$ is a subobject $S \subseteq M \times I \times I$ such that the internal quiver induced by
    \begin{equation*}
        S \hookrightarrow M \times I \times I \rightrightarrows I
    \end{equation*}
    has bounded depth, as in~\cref{def:internal_quiver_bounded_depth}.
\end{definition}
% Let $\Cat$ be a Cartesian category, and let $T$ be a comonad on $\Cat$. A {\em nilpotent constraint} on $T$ is a nilpotent Cartesian endofunctor $F \colon \Cat \rightarrow \Cat$ together with a natural transformation $\eta \colon T \rightarrow F$.

This is sufficient to give a notion of locality.



\subsection{The category of coalgebras}

Let $\Cat$ be a Cartesian closed category, and let $F\colon \Cat \rightarrow \Cat$ be a nilpotent architecture. We can consider $\FCoalg$, the category of coalgebras over $F$ \pietro{add ref}. Objects are simply maps
\begin{equation*}
    X \rightarrow F X,
\end{equation*}
where $X \in \Obj(\Cat)$. Morphisms are given by commutative diagrams
\begin{equation*}
    \begin{tikzcd}
        X \arrow[rightarrow]{d} \arrow[rightarrow]{r}{\phi}
        & Y \arrow[rightarrow]{d} \\
        FX \arrow[rightarrow]{r}{F \phi}
        & FY
    \end{tikzcd}
\end{equation*}
As $F$ is Cartesian, the category $\FCoalg$ is also Cartesian, with products given by
\begin{equation*}
    \prod_{i \in I} X_i \rightarrow \prod_{i \in I} F X_i \simeq F \prod_{i \in I} X_i.
\end{equation*}
As $F$ is nilpotent and preserves finite products, it is straightforward to verify that the forgetful functor
\begin{equation*}
    U \colon \FCoalg \rightarrow \Cat
\end{equation*}
has a right adjoint $\Phi$, called the cofree functor, given by
\begin{equation*}
    \Phi(X) = X \times F X \times \dots \times F^{n-1} X \rightarrow F X \times \dots \times F^{n-1} X.
\end{equation*}

\begin{remark}
    This adjunction is comonadic \pietro{explain! cite Beck's theorem!}, and $\FCoalg$ is the Eilenberg-Moore category $\Cat^T$, where $T = U\Phi$.
\end{remark}

\section{Machines}

\begin{definition}\label{def:machine}
    Let $\Cat$ be a category, equipped with a terminal object $\pt$. Let $X \in \Obj(\Cat)$. A map
    \begin{equation*}
        \varrho \colon X \rightarrow X
    \end{equation*}
    is a {\em machine} if the induced transformation
    \begin{equation*}
        \Hom_\Cat(\pt, X) \rightarrow \Hom_\Cat(\pt, X)
    \end{equation*}
    has a unique fixed point $S_\varrho$, which we call the {\em stable state} of $\varrho$.
\end{definition}

\pietro{Fare qualche esempio, tipo contrazioni.}

\begin{definition}\label{def:parametric_machine}
    Let $\Cat$ be a category with finite products. A {\em parametric machine} is simply a machine on the Kleisli category for the comonad $P \times \anon$, for some $P \in \Obj(\Cat)$. More explicitly, a map
    \begin{equation*}
        \varrho \colon P \times X \rightarrow X
    \end{equation*}
    is a {\em machine} if there exists a unique map $S_\varrho$ such that the following diagram commutes
    \begin{equation}
        \begin{tikzcd}[column sep=large]\label[diagram]{diag:machine_condition}
            P \arrow[rightarrow]{r}{\left(\id, S_\varrho\right)}
            \arrow[rightarrow,swap]{dr}{S_\varrho}
            & P \times X \arrow[rightarrow]{d}{\varrho} \\
            & X
        \end{tikzcd}
    \end{equation}
\end{definition}

\Cref{diag:machine_condition} is analogous to the machine condition in~\cite{2020arXiv200702777V} in the case when $X$ is an Abelian group and $\varrho\colon P \times X \rightarrow X$ is the sum of a map $P \rightarrow X$ and a map $X \rightarrow X$. This approach differs from the one taken in~\cite{2020arXiv200702777V}, in that here the object $P$ represents both input space and parameters. In what follows, we will show how to construct examples of {\em machines} using {\em nilpotent functors}.

\begin{theorem}\label{thm:nilpotent_architecture}
    Let $F\colon \Cat \rightarrow \Cat$ be a nilpotent endofunctor. Let $\Phi X$ be a cofree coalgebra in $\FCoalg$. Let $P$ be a $F$ coalgebra. Then a map $P \times F X \rightarrow X$ induces a parametric machine $P \times \Phi X \rightarrow \Phi X$.
\end{theorem}

\begin{proof}
    \pietro{TODO per induzione.}
\end{proof}

\pietro{La sfida principale qui è trovare un buona architettura $F$ ma anche una buona coalgebra $P$. Probabilmente da menzionare da qualche parte che se la categoria di base è buona, $\FCoalg$ ha molte buone proprietà.}

\section{Recurrent and convolutional neural networks}

\pietro{discuss shape of filter?}

\section{Novel architectures}

\bibliographystyle{abbrv}
\bibliography{References}

\end{document}
