\documentclass[12pt]{article}

\usepackage{amsmath, amsthm, amssymb}
\usepackage{amssymb}
\usepackage{bbm}

\usepackage[hidelinks]{hyperref}
\usepackage{cleveref}
\usepackage{url}

%%% For figures
\usepackage{graphicx}
\DeclareGraphicsExtensions{.pdf,.png,.jpg}

%%% For tables
\usepackage[table]{xcolor}

%specifying table and figure packages
\usepackage{caption}

\usepackage[T1]{fontenc}
\usepackage[all]{xy}
\usepackage[inline]{enumitem}

\newtheorem{theorem}{Theorem}
\newtheorem{corollary}{Corollary}
\newtheorem{remark}{Remark}
\newtheorem{example}{Example}
\newtheorem{definition}{Definition}
\newtheorem{proposition}{Proposition}
\newtheorem{lemma}{Lemma}

\newcommand{\mattia}[1]{\textcolor{cyan}{#1}}
\newcommand{\pietro}[1]{\textcolor{teal}{#1}}
\newcommand{\Ban}{{\mathbf{Ban}}}
\newcommand{\Top}{{\mathbf{Top}}}
\newcommand{\forget}{{\textnormal{forget}}}
\newcommand{\free}{{\textnormal{free}}}
\newcommand{\pt}{{\textnormal{pt}}}
\newcommand{\Hom}{{\textnormal{Hom}}}
\newcommand{\End}{{\textnormal{End}}}
\newcommand{\Fun}{{\textnormal{Fun}}}
\newcommand{\Aut}{{\textnormal{Aut}}}
\newcommand{\Obj}{{\textnormal{Obj}}}
\newcommand{\id}{{\textnormal{id}}}
\newcommand{\Morph}{{\textnormal{Morph}}}
\newcommand{\Set}{{\mathbf{Set}}}
\newcommand{\Cat}{{\mathbf{C}}}
\newcommand{\DCat}{{\mathbf{D}}}
\newcommand{\ICat}{{\mathbf{I}}}
\newcommand{\JCat}{{\mathbf{J}}}
\newcommand{\LCat}{{\mathbf{L}}}
\newcommand{\range}[2]{{\{{#1}, \dots,{#2}\}}}
\newcommand{\anon}{{\,\mbox{-}\,}}

\crefname{diagram}{diag.}{diags.}
\Crefname{diagram}{Diagram}{Diagrams}
\creflabelformat{diagram}{#2(#1)#3}

\title{Comonadic neural network architectures}
\author{
    Pietro Vertechi \and Mattia G. Bergomi
}
\date{}
\begin{document}
\maketitle
\begin{abstract}
\end{abstract}

\section{Architecture}

We begin by giving an abstract definition of {\em architecture}. \pietro{Non so se vogliamo questa come definizione, oppure se prendere come definizione le ipotesi di \cref{thm:kan_architecture}.}

\begin{definition}\label{def:architecture}
    An {\em architecture} is a natural transformation from a comonad to an endofunctor. More explicitly, given a category $\DCat$, an {\em architecture} on $\DCat$ is composed of the following data:
    \begin{itemize}
        \item a comonad $T$ on $\DCat$,
        \item an endofunctor $S \colon \DCat \rightarrow \DCat$,
        \item a natural transformation $\eta \colon T \rightarrow S$.
    \end{itemize}
\end{definition}

% An architecture $(\DCat, T, S, \eta)$ is {\em valid} if the following conditions are met:
% \begin{itemize}
%     \item $\DCat$ has finite products,
%     \item for all $(X, +)$ commutative monoid in $\DCat$ \pietro{meglio introdurre la categoria lineare a parte}, and for all
%           \begin{equation*}
%               f \colon SX \rightarrow X,
%           \end{equation*}
%           the composition
%           \begin{equation*}
%               f \circ \eta_X \colon TX \rightarrow X
%           \end{equation*}
%           is a {\em machine}~\cite{2020arXiv200702777V} in the Kleisli category $\DCat_T$.
% \end{itemize}

\pietro{Da discutere il caso continuo (interpretazione comonadica di macchine di Volterra). Probabilmente da fare con enriched category theory.}

This is sufficient to give notions of composability, equivariance, and locality. In the reminder of the section, we will first give a broad class of architectures. Then, we will consider a simpler special case and show that it is sufficient to recover, among other examples, classical recurrent and convolutional neural networks.

Option: drop this and do it for a diagram
\begin{equation*}
    M \rightarrow C_1 \rightrightarrows C_0.
\end{equation*}

\begin{theorem}\label{thm:kan_architecture}
    Let $p\colon \ICat \rightarrow \JCat$ be a functor between small categories. Let $H$ be a subfunctor of the composition
    \begin{equation*}
        \ICat^{op} \times \ICat \xrightarrow{p \times p}
        \JCat^{op} \times \JCat \xrightarrow{\Hom} \Set.
    \end{equation*}
    Let $\Cat$ be a complete category. Then, $p$ and $H$ induce an architecture on $\Cat^\ICat$, which we call a {\em Kan architecture}.
\end{theorem}

\begin{proof}
    $p$ induces a functor $p^* \colon \Cat^\JCat \rightarrow \Cat^\ICat$, given by pre-composition. As $\Cat$ is complete, $p^*$ admits a right adjoint $p_* \colon \Cat^\JCat \rightarrow \Cat^\ICat$, its {\em right Kan extension}~\cite[Chapt.~X]{mac2013categories}. This adjunction $p^* \dashv p_*$ induces a comonad
    \begin{equation*}
        T := p^*p_* \colon \Cat^\ICat \rightarrow \Cat^\ICat.
    \end{equation*}
    Given $i \in \Obj(\ICat)$, $T(i)$ can be described explicitly using the {\em end} formula,~\cite[Eq.~X.4.3]{mac2013categories}, which we report here in our notation for a functor $F \colon \ICat \rightarrow \Cat$:
    \begin{equation}\label{eq:endformula}
        p_* F = \int_i F(i)^{\Hom(\anon, p(i))}.
    \end{equation}
    The integral represents an {\em end}~\cite[Sect.~IX.5]{mac2013categories} over the category $\ICat$. From~\cref{eq:endformula} we can deduce that,
    \begin{equation}\label{eq:comonad_endformula}
        T F = \int_i F(i)^{\Hom(p(\anon), p(i))}.
    \end{equation}
    We then define
    \begin{equation}\label{eq:functor_endformula}
        S F := \int_i F(i)^{H(\anon, i)},
    \end{equation}
    where $S F$ is a functor, as $H$ is closed by pre-composition of morphisms in the image of $p$. By functoriality of end, the inclusion $H \hookrightarrow \Hom \circ p \times p$ gives us a natural transformation
    \begin{equation*}
        \eta \colon T \rightarrow S.
    \end{equation*}
\end{proof}

\subsection{Simple Kan architecture}

\Cref{thm:kan_architecture} gives an explicit example of architecture that is still somewhat complex. Let us consider a particular case. Let  $p\colon J \hookrightarrow \JCat$ be the underlying discrete subcategory, having the same objects as $\JCat$, but only identity morphisms. In this scenario, \cref{thm:kan_architecture} simplifies greatly. In particular, $H(i, j)$ can simply be written as $\Hom_\JCat(i, j) \cap M$, for some $M \subseteq \Morph(\JCat)$. Furthermore, ends over discrete categories are just products, so~\cref{eq:comonad_endformula,eq:functor_endformula} become:
\begin{equation}\label{eq:productformula}
    T G = \prod_{j \in J} G(j)^{\Hom_\JCat(\anon, j)}
    \quad\text{ and }\quad
    S G = \prod_{j \in J} G(j)^{Hom_\JCat(\anon, j) \cap M}.
\end{equation}
This result is summarized in the following theorem.

\begin{theorem}\label{thm:discrete_kan_architecture}
    Let $\JCat$ be a small category, and let $J = \Obj(\JCat)$ be its underlying discrete category. Let $M \subseteq \Morph(\JCat)$ be a subset of morphisms in $\JCat$. Let $\Cat$ be a category with small products. Then $\JCat$ and $M$ induce an architecture on $\Cat^J$.
\end{theorem}

\begin{proof}
    The comonad and functor are given by~\cref{eq:productformula}.
\end{proof}

\section{Comonadic machines}

Show how it's possible to obtain a machine from an architecture.

\section{Compactly-generated Hausdorff spaces}



\section{Recurrent and convolutional neural networks}

Monoids can be considered as categories with one object. Let us consider the category corresponding to $\mathbb N$. Let $M$ be the set containing only the generator morphism. Then the above construction recovers recurrent neural networks. Analogously, considering $\mathbb N \times \mathbb N$ we obtain convolutional neural networks, where $M$ denotes the shape of the filter.
\pietro{TODO: mention 2 objects for conv, and strides!}
\pietro{Maybe also do functors, to generalize GENEO? Somehow discuss continuous version?}

\section{Novel architectures}

Here, we explore the scenario where the category $\Cat$ has more than one object.
\pietro{TODO: figure out a good application.}
\pietro{Mention partial symmetries because of non-composability.}

\bibliographystyle{abbrv}
\bibliography{References}

\end{document}
