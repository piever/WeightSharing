\documentclass[12pt]{article}

\usepackage{amsmath, amsthm, amssymb}
\usepackage{amssymb}
\usepackage{bbm}

\usepackage[hidelinks]{hyperref}
\usepackage{cleveref}
\usepackage{url}

%%% For figures
\usepackage{graphicx}
\DeclareGraphicsExtensions{.pdf,.png,.jpg}

%%% For tables
\usepackage[table]{xcolor}

%specifying table and figure packages
\usepackage{caption}

\usepackage[T1]{fontenc}
\usepackage[all]{xy}
\usepackage[inline]{enumitem}

\newtheorem{theorem}{Theorem}
\newtheorem{corollary}{Corollary}
\newtheorem{remark}{Remark}
\newtheorem{example}{Example}
\newtheorem{definition}{Definition}
\newtheorem{proposition}{Proposition}
\newtheorem{lemma}{Lemma}

\newcommand{\mattia}[1]{\textcolor{cyan}{#1}}
\newcommand{\pietro}[1]{\textcolor{teal}{#1}}
\newcommand{\Ban}{{\mathbf{Ban}}}
\newcommand{\Top}{{\mathbf{Top}}}
\newcommand{\pt}{{\textnormal{pt}}}
\newcommand{\Hom}{{\textnormal{Hom}}}
\newcommand{\End}{{\textnormal{End}}}
\newcommand{\Fun}{{\textnormal{Fun}}}
\newcommand{\Aut}{{\textnormal{Aut}}}
\newcommand{\Obj}{{\textnormal{Obj}}}
\newcommand{\id}{{\textnormal{id}}}
\newcommand{\Morph}{{\textnormal{Morph}}}
\newcommand{\Set}{{\mathbf{Set}}}
\newcommand{\Cat}{{\mathbf{C}}}
\newcommand{\DCat}{{\mathbf{D}}}
\newcommand{\ICat}{{\mathcal{I}}}
\newcommand{\JCat}{{\mathcal{J}}}
\newcommand{\LCat}{{\mathbf{L}}}
\newcommand{\QCat}{{\mathbf{Q}}}
\newcommand{\range}[2]{{\{{#1}, \dots,{#2}\}}}
\newcommand{\anon}{{\,\mbox{-}\,}}

\crefname{diagram}{diag.}{diags.}
\Crefname{diagram}{Diagram}{Diagrams}
\creflabelformat{diagram}{#2(#1)#3}

\title{Weight sharing via comonads}
\author{
    Pietro Vertechi \and Mattia G. Bergomi
}
\date{}
\begin{document}
\maketitle
\begin{abstract}
\end{abstract}

\section{Architecture}

We begin by giving an abstract definition of {\em architecture}.

\begin{definition}\label{def:architecture}
    Let $(\Cat, \otimes)$ be a closed monoidal category.
    An {\em architecture} on $(\Cat, \otimes)$ is composed of the following data:
    \begin{itemize}
        \item a monoidal comonad $T$ on $\Cat$,
        \item an endofunctor $S \colon \Cat \rightarrow \Cat$,
        \item a natural transformation $\eta \colon T \rightarrow S$.
    \end{itemize}
\end{definition}

% \begin{definition}\label{def:locality_constraint}
%     A {\em locality constraint} on a an architecture $T$ is a natural transformation from $T$ to an endofunctor. More explicitly, given an architecture $T\colon \Cat \rightarrow \Cat$, a {\em locality constraint} is an endofunctor $S \colon \Cat \rightarrow \Cat$ together with a natural transformation $\eta \colon T \rightarrow S$.
% \end{definition}

% \begin{definition}\label{def:validarchitecture}
%     Let us assume that $\Cat$ has finite products, and let $\iota \colon \LCat \rightarrow \Cat$ be a product-preserving forgetful functor, with $\LCat$ an additive category.
%     We say that an architecture is {\em valid} if, for all $X = \iota(M)$, with $M \in \Obj(\LCat)$, and for all $f \colon SX \rightarrow X$, the composition $f \circ \eta_X \colon TX \rightarrow X$ is a {\em machine}~\cite{2020arXiv200702777V} in the Kleisli category $\Cat_T$.
% \end{definition}

\pietro{Da discutere il caso continuo (interpretazione comonadica di macchine di Volterra). Probabilmente da fare con enriched category theory.}
This is sufficient to give notions of composability, weight sharing, and locality. In the reminder of the section, we will first give a broad class of architectures. Then, we will consider a simpler special case and show that it is sufficient to recover, among other examples, classical recurrent and convolutional neural networks.

\subsection{Examples}

\pietro{Da semplificare un pochino, e iniziare con un esempio più semplice e concreto.}

\begin{theorem}\label{thm:kan_architecture}
    Let $p\colon \ICat \rightarrow \JCat$ be a functor between small categories. Let $H$ be a subfunctor of the composition
    \begin{equation*}
        \ICat^{op} \times \ICat \xrightarrow{p \times p}
        \JCat^{op} \times \JCat \xrightarrow{\Hom} \Set.
    \end{equation*}
    Let $\Cat$ be a complete category. Then, $p$ and $H$ induce an architecture on $\Cat^\ICat$, which we call a {\em Kan architecture}.
\end{theorem}

\begin{proof}
    $p$ induces a functor $p^* \colon \Cat^\JCat \rightarrow \Cat^\ICat$, given by pre-composition. As $\Cat$ is complete, $p^*$ admits a right adjoint $p_* \colon \Cat^\JCat \rightarrow \Cat^\ICat$, its {\em right Kan extension}~\cite[Chapt.~X]{mac2013categories}. This adjunction $p^* \dashv p_*$ induces a comonad
    \begin{equation*}
        T := p^*p_* \colon \Cat^\ICat \rightarrow \Cat^\ICat.
    \end{equation*}
    Given $i \in \Obj(\ICat)$, $T(i)$ can be described explicitly using the {\em end} formula,~\cite[Eq.~X.4.3]{mac2013categories}, which we report here in our notation for a functor $F \colon \ICat \rightarrow \Cat$:
    \begin{equation}\label{eq:endformula}
        p_* F = \int_i F(i)^{\Hom(\anon, p(i))}.
    \end{equation}
    The integral represents an {\em end}~\cite[Sect.~IX.5]{mac2013categories} over the category $\ICat$. From~\cref{eq:endformula} we can deduce that,
    \begin{equation}\label{eq:comonad_endformula}
        T F = \int_i F(i)^{\Hom(p(\anon), p(i))}.
    \end{equation}
    We then define
    \begin{equation}\label{eq:functor_endformula}
        S F := \int_i F(i)^{H(\anon, i)},
    \end{equation}
    where $S F$ is a functor, as $H$ is closed by pre-composition of morphisms in the image of $p$. By functoriality of end, the inclusion $H \hookrightarrow \Hom \circ p \times p$ gives us a natural transformation
    \begin{equation*}
        \eta \colon T \rightarrow S.
    \end{equation*}
\end{proof}

\Cref{thm:kan_architecture} gives an explicit example of architecture that is still somewhat complex. Let us consider a particular case. Let  $p\colon J \hookrightarrow \JCat$ be the underlying discrete subcategory, having the same objects as $\JCat$, but only identity morphisms. In this scenario, \cref{thm:kan_architecture} simplifies greatly. In particular, $H(i, j)$ can simply be written as $\Hom_\JCat(i, j) \cap M$, for some $M \subseteq \Morph(\JCat)$. Furthermore, ends over discrete categories are just products, so~\cref{eq:comonad_endformula,eq:functor_endformula} become:
\begin{equation}\label{eq:productformula}
    T G = \prod_{j \in J} G(j)^{\Hom_\JCat(\anon, j)}
    \quad\text{ and }\quad
    S G = \prod_{j \in J} G(j)^{Hom_\JCat(\anon, j) \cap M}.
\end{equation}
This result is summarized in the following theorem.

\begin{theorem}\label{thm:discrete_kan_architecture}
    Let $\JCat$ be a small category, and let $J = \Obj(\JCat)$ be its underlying discrete category. Let $M \subseteq \Morph(\JCat)$ be a subset of morphisms in $\JCat$. Let $\Cat$ be a category with small products. Then $\JCat$ and $M$ induce an architecture on $\Cat^J$.
\end{theorem}

\begin{proof}
    The comonad and functor are given by~\cref{eq:productformula}.
\end{proof}

\pietro{Questo teorema e' noto ma credo valga la pena menzionarlo.}

\begin{theorem}\label{thm:categorical_architecture}
    Let $\QCat$ be a locally Cartesian closed category. Then, there is a well defined notion of category {\em internal} to $\QCat$~\cite{mac2013categories}.
    Let $\JCat = s, t \colon C_1 \rightarrow C_0$ be such an internal category. $\JCat$ induces an architecture
    \begin{equation*}
        s_!t^*\colon \QCat/C_0 \rightarrow \QCat/C_0.
    \end{equation*}
\end{theorem}

\begin{proof}
    The map $t_*s^*\colon \QCat/C_0 \rightarrow \QCat/C_0$ is a monad~\cite[Thm.~V.8.2]{Mac_Lane_1994}. As $\QCat$ is locally Cartesian closed, $s^*$ has a right adjoint $s_!$, hence we can consider $s_!t^*$. As $T = s_!t^*$ is the right adjoint of the monad $t_*s^*$, it is a comonad. Being a right adjoint, $T$ preserves finite products.
\end{proof}

\pietro{Discutere ultima comonade di \cite{uustalu_comonadic_2008} come caso particolare.}

\section{Weight sharing}

One of the key reasons for the success of neural networks is the notion of {\em weight sharing}. In its simplest form, one can consider, among maps $[M, X] \rightarrow [M, X]$, those that originate from maps $[M, X] \rightarrow X$. Indeed, given
\begin{equation*}
    {\cal F} \colon [M, X] \rightarrow X,
\end{equation*}
one can consider the map
\begin{align*}
    \widehat{\cal F} \colon [M, X] &\rightarrow [M, X] \\
    \left(\widehat{\cal F}\phi\right)(m) &= {\cal F}(a \mapsto \phi(ma)).
\end{align*}

This simple notion of weight sharing does not seem sufficient to cover complex neural networks, made up of several nodes, each with their own symmetries, which interact with each other in non-trivial ways.

We aim to show that the simple notion of weight sharing defined above is sufficient to recover, among other examples, classical recurrent and convolutional neural networks. However, to do so, we will need to generalize not the weight sharing strategy, but rather the underlying category on which such strategy takes place.

Here, we consider how to define a notion of weight sharing given an architecture $T\colon\Cat \rightarrow \Cat$. Let $T$ be an architecture on a closed monoidal category $(\Cat, \otimes)$. Let $\Cat^T$ be Eilenberg-Moore category for the comonad $T$. It is a known fact~\cite{Moerdijk_2002,pastro2009closed} that the monoidal structure on $\Cat$ lifts to a monoidal structure on $\Cat^T$ in such a way that the forgetful functor $U\colon \Cat^T \rightarrow \Cat$ is strict monoidal.

\begin{proposition}\label{prop:cofree_exponentiating}
    Let $T\colon \Cat \rightarrow \Cat$ be an architecture. Then cofree coalgebras in $\Cat^T$ are exponentiating. In particular, let $F\colon \Cat \rightarrow \Cat^T$ denote the cofree functor. Then, there exists a natural isomorphism
    \begin{equation*}
        \Hom(A\otimes B, FX) \simeq \Hom(A, F[UB, X]).
    \end{equation*}
\end{proposition}

\begin{proof}
    By a straightforward computation:
    \begin{align*}
        \Hom_{\Cat^T}(A\otimes B, FX)
        &\simeq \Hom_{\Cat}(U(A \otimes B), X)\\
        &\simeq \Hom_{\Cat}(UA \otimes UB, X)\\
        &\simeq \Hom_{\Cat}(UA, [UB, X])\\
        &\simeq \Hom_{\Cat^T}(A, F[UB, X]).
    \end{align*}
\end{proof}

\begin{theorem}\label{thm:weight_sharing}
    Let $M \in \Cat^T$ be a monoid. Then $M$ induces a comonad:
    \begin{equation*}
        X \mapsto T[UM, X].
    \end{equation*}
\end{theorem}

\begin{proof}
    \pietro{Use $F[UM, X]$ and Kleisli category. Attenzione: Kleisli non ha per forza un prodotto tensore.} Monoidal functor
    \begin{equation*}
        \left(\Cat^T\right)^{op} \rightarrow \Fun(\Cat_T, \Cat_T),
    \end{equation*}
    hence monoid goes to comonad. That, and free forget from Kleisli to $\Cat$.
\end{proof}

\section{Locality}

\pietro{Fai la localita' in termini di sottoggetti di M o qualcosa di simile. Funziona?}

\section{Comonadic machines}

Show how it's possible to obtain a machine from an architecture.

\section{Recurrent and convolutional neural networks}

Monoids can be considered as categories with one object. Let us consider the category corresponding to $\mathbb N$. Let $M$ be the set containing only the generator morphism. Then the above construction recovers recurrent neural networks. Analogously, considering $\mathbb N \times \mathbb N$ we obtain convolutional neural networks, where $M$ denotes the shape of the filter.
\pietro{TODO: mention 2 objects for conv, and strides!}
\pietro{Maybe also do functors, to generalize GENEO? Somehow discuss continuous version?}

\section{Novel architectures}

Here, we explore the scenario where the category $\Cat$ has more than one object.
\pietro{TODO: figure out a good application.}
\pietro{Mention partial symmetries because of non-composability.}

\bibliographystyle{abbrv}
\bibliography{References}

\end{document}
