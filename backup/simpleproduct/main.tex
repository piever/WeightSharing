\documentclass[12pt]{article}

\usepackage{amsmath, amsthm, amssymb}
\usepackage{amssymb}
\usepackage{bbm}

\usepackage[hidelinks]{hyperref}
\usepackage{cleveref}
\usepackage{url}

%%% For figures
\usepackage{graphicx}
\DeclareGraphicsExtensions{.pdf,.png,.jpg}

%%% For tables
\usepackage[table]{xcolor}

%specifying table and figure packages
\usepackage{caption}

\usepackage[T1]{fontenc}
\usepackage[all]{xy}
\usepackage[inline]{enumitem}

\newtheorem{theorem}{Theorem}
\newtheorem{corollary}{Corollary}
\newtheorem{remark}{Remark}
\newtheorem{example}{Example}
\newtheorem{definition}{Definition}
\newtheorem{proposition}{Proposition}
\newtheorem{lemma}{Lemma}

\newcommand{\mattia}[1]{\textcolor{cyan}{#1}}
\newcommand{\pietro}[1]{\textcolor{teal}{#1}}
\renewcommand{\P}{{\mathcal{P}}}
\newcommand{\Ban}{{\mathbf{Ban}}}
\newcommand{\Top}{{\mathbf{Top}}}
\newcommand{\forget}{{\textnormal{forget}}}
\newcommand{\free}{{\textnormal{free}}}
\newcommand{\pt}{{\textnormal{pt}}}
\newcommand{\Hom}{{\textnormal{Hom}}}
\newcommand{\End}{{\textnormal{End}}}
\newcommand{\Aut}{{\textnormal{Aut}}}
\newcommand{\Obj}{{\textnormal{Obj}}}
\newcommand{\id}{{\textnormal{id}}}
\newcommand{\Morph}{{\textnormal{Morph}}}
\newcommand{\Cat}{{\mathbf{C}}}
\newcommand{\LCat}{{\mathbf{L}}}
\newcommand{\range}[2]{{\{{#1}, \dots,{#2}\}}}
\newcommand{\anon}{{\,\mbox{-}\,}}
\newcommand{\causal}{{\Delta_{\P, \Cat}}}

\crefname{diagram}{diag.}{diags.}
\Crefname{diagram}{Diagram}{Diagrams}
\creflabelformat{diagram}{#2(#1)#3}

\title{
    Causal machines
    }
\author{
    Pietro Vertechi \and Mattia G. Bergomi
}
\date{}
\begin{document}
    \maketitle
    \begin{abstract}
    \end{abstract}

    \section{Causal category}
    Let $\Cat$ be a category with finite products. Let $\P = (P, \preceq)$ be a finite preordered set. Let us consider the functor category $[P, \Cat]$, where $P$ is seen as a discrete category. We recall, and slightly adapt, a definition of comonad originally given in~\cite{uustaluComonadicNotionsComputation2008}.
    \begin{definition}\label{def:causalcomonad}
        The {\em causal comonad} $D\colon [P, \Cat] \rightarrow [P, \Cat]$ is given by
        \begin{equation*}
            (DA)_p = \prod_{q \preceq p} A_q.
        \end{equation*}
        $D$ is equipped with a counit $\epsilon\colon D \rightarrow \id$ given by the projection $\pi_p$:
        \begin{equation*}
            \epsilon_A \colon \prod_{q \preceq p} A_q \rightarrow A_p.
        \end{equation*}
        It is also equipped with a comultiplication $\delta\colon D \rightarrow DD$:
        \begin{equation*}
            \delta_A \colon \prod_{q \preceq p} A_q \rightarrow \prod_{q \preceq p} \left(\prod_{r \preceq q} A_r\right),
        \end{equation*}
        induced by the projections $\prod_{q \preceq p} A_q \rightarrow \prod_{r \preceq q} A_r$.
    \end{definition}
    \begin{definition}\label{def:causalcategory}
        The {\em causal category} $\causal$ is the Kleisli category associated to the causal comonad $D$.
    \end{definition}

    While \cref{def:causalcomonad,def:causalcategory} seem abstract, it is possible to describe the causal category $\causal$ explicitly.
    Objects in $\causal$ are collections of objects of $\Cat$ indexed by $P$, that is $\{A_p\}_{p\in P}$. Let us consider two such objects $\{A_p\}_{p\in P}, \{B_p\}_{p\in P} \in \Obj(\causal)$. Then,
    \begin{equation*}
        \Hom_\causal\left(\{A_p\}_{p\in P}, \{B_p\}_{p\in P}\right) =
        \prod_{p \in P} \Hom_\Cat\left(\prod_{q \preceq p} A_q, B_p\right).
    \end{equation*}

    \section{Stream comonad in Cartesian closed categories}
    Here, we recall the notion of {\em stream comonad}~\cite{uustaluComonadicNotionsComputation2008} in a general Cartesian closed category.
    Let $\Cat$ be a Cartesian closed category. $\Cat$ is a monoidal category, with monoidal structure given by the Cartesian product. Let us consider the functor category
    \begin{equation*}
        \EndCat.
    \end{equation*}
    $\EndCat$ is also a monoidal category, with monoidal structure given by the composition of functors. There is a functor
    \begin{align*}
        \Cat^{op} &\rightarrow \EndCat \\
        M &\mapsto [M, -],
    \end{align*}
    which preserves the monoidal structure, as $[M \times N, -]$ is naturally isomorphic to $[M, [N, -]]$, for $M, N \in \Obj(\Cat)$.
    As a consequence, whenever $M \in \Obj(\Cat)$ is a monoid, $[M, -]$ is a comonad.
    \begin{definition}\label{def:streamcomonad}\cite[Sect.~3.1]{uustaluComonadicNotionsComputation2008}
        Let $M$ be a monoid in a Cartesian closed category $\Cat$.
        The {\em stream comonad} $D_M\colon \Cat \rightarrow \Cat$ is given by
        \begin{equation*}
            D_M A = [M, A]
        \end{equation*}
        The {\em stream category} $\Stream$ is the Kleisli category associated to the stream comonad $D_M$.
    \end{definition}

    In the following sections, we will focus on $\StreamN$ and $\StreamR$, which correspond to discrete-time and continuous-time recurrent architectures.

        % \section{Categorical preliminaries}

    % Let $\Cat$ be a category with small products, and let $\JCat$ be a small category.
    % The aim of this section is to describe concretely a specific functor
    % \begin{equation*}
    %    \Cat \rightarrow \Fun(\JCat, \Cat).
    % \end{equation*}
    % Given $j\in\Obj(\JCat)$, we can consider the functor
    % \begin{align*}
    %     \Phi_j &\colon \Cat \rightarrow \FunCat\\
    %     \Phi_j(X) &:= i \mapsto X ^ {\Hom_\JCat(i, j)}.
    % \end{align*}
    % Multiplying all these functors together, we obtain
    % \begin{equation*}
    %     \Phi := \prod_{j \in \Obj(\JCat)} \Phi_j \colon \Cat \rightarrow \FunCat.
    % \end{equation*}
    % Let $X, Y\in\Obj(\Cat)$. It is possible to describe $\Hom_{\FunCat}(\Phi(X), \Phi(Y))$ explicitly.
    
    % \begin{lemma}\label{lm:yoneda_prod}
    %     Let $F \in \FunCat$, $j \in \JCat$, and $Y \in \Obj(\Cat)$. Then,
    %     \begin{equation*}
    %         \Hom_{\FunCat}(F, \Phi_j(Y)) \simeq \Hom_\Cat(F(j), Y).
    %     \end{equation*} 
    % \end{lemma}
    % \begin{proof}
    %     \begin{align*}
    %         \Hom_{\FunCat}(F, \Phi_j(Y))
    %         &\simeq \Hom_{\FunCat}\left(F, Y ^ {\Hom_\JCat(\anon, j)}\right) \\
    %         &\simeq \Hom_{\PreSheaves}\left(\Hom_\JCat(\anon, j), \Hom_\Cat(F(\anon), Y)\right) \\
    %         &\simeq \Hom_\Cat(F(j), Y),
    %     \end{align*}
    %     where the last isomorphism is the Yoneda lemma.
    % \end{proof}

    % \begin{lemma}\label{lm:yoneda_all}
    %     Let $F \in \FunCat$, $j \in \JCat$, and $Y \in \Obj(\Cat)$. Then,
    %     \begin{equation*}
    %         \Hom_{\FunCat}(F, \Phi(Y)) \simeq \prod_{j\in \Obj(\JCat)}\Hom_\Cat(F(j), Y).
    %     \end{equation*} 
    % \end{lemma}
    % \begin{proof}
    %     By~\cref{lm:yoneda_prod}.
    % \end{proof}
    
    % \begin{proposition}\label{prop:morphisms}
    %     Let $X, Y \in \Obj(\Cat)$. Then         
    %     \begin{equation*}
    %         \Hom_{\FunCat}(\Phi(X), \Phi(Y)) \simeq
    %         \prod_{j\in \Obj(\JCat)}\Hom_\Cat\left(X^{out(j)}, Y\right),
    %     \end{equation*}
    %     where $out(j)$ is the set of morphisms whose domain is $j$.
    % \end{proposition}
    % \begin{proof}
    %     By~\cref{lm:yoneda_all}.
    % \end{proof}

    % \begin{proposition}\label{prop:product}
    %     $\Phi$ preserves small products.
    % \end{proposition}
    % \begin{proof}
    %     \pietro{TODO}
    % \end{proof}
  
    % \begin{remark}
    %     Discuss comonad approach from \cite{Ahman_2016}.
    % \end{remark}

    % \subsection{Recurrent neural networks}

    % Let us consider the category $\mathbf N$, with one object whose endomorphisms are given by the monoid $\mathbb N$. Then the above construction recovers recurrent neural networks.

    \section{Categorical preliminaries}

    Let $\Cat$ be a cocomplete Cartesian closed category, and let $\PCat$ be a small category.
    The aim of this section is to describe how a presheaf $P$ on $\PCat$ induces an adjunction
    \begin{align*}
        L_P &\colon \Fun(\PCat, \Cat) \rightarrow \Cat,\\
        R_P &\colon \Cat \rightarrow \Fun(\PCat, \Cat).
    \end{align*}
    Let $\PreSheaves$ denote the category of presheaves on $\PCat$. Let us define $R$ as follows.
    \begin{align*}
        R \colon \PreSheaves^{op} &\rightarrow \Fun(\Cat, \Fun(\PCat, \Cat)),\\
        R_P &:= X \mapsto X^{P(\anon)}.
    \end{align*}
    Let us consider:
    \begin{align*}
        L\colon \PCat &\rightarrow \Fun(\Fun(\PCat, \Cat), \Cat)\\
        L_p &:= X \mapsto X(p).
    \end{align*}
    As $\Cat$ has small colimits, $L$ extends uniquely (up to natural isomorphism) to a colimit-preserving functor
    \begin{equation*}
        L\colon \PreSheaves \rightarrow \Fun(\Fun(\PCat, \Cat), \Cat).
    \end{equation*}

    \begin{proposition}\label{prop:adjunction}
        For all $P \in \PreSheaves$, $L_P$ is the left adjoint of $R_P$.
    \end{proposition}
    \begin{proof}
        Let $X \in \Obj(\FunCat)$, $Y \in \Obj(\Cat)$. We wish to prove that
        \begin{equation*}
            \Hom_{\FunCat}(X, R_P(Y)) \simeq \Hom_\Cat(L_P, Y).
        \end{equation*}
        Thanks to the density theorem~\cite[Chapt.~3.7]{mac2013categories}, we can write $P$ as a colimit of representable presheaves:
        \begin{equation*}
            P \simeq \varinjlim \Hom(\anon, p).
        \end{equation*}
        Then,
        \begin{align*}
            \Hom_{\FunCat}(X, R_P(Y))
            &\simeq \Hom_{\FunCat}\left(X, Y ^ {P(\anon)}\right) \\
            &\simeq \Hom_{\PreSheaves}\left(P(\anon), \Hom_\Cat(X(\anon), Y)\right) \\
            &\simeq \Hom_{\PreSheaves}\left(\varinjlim \Hom(\anon, p), \Hom_\Cat(X(\anon), Y)\right) \\
            &\simeq \varprojlim\Hom_{\PreSheaves}\left(\Hom(\anon, p), \Hom_\Cat(X(\anon), Y)\right) \\
            &\simeq \varprojlim\Hom_\Cat(X(p), Y)\\
            &\simeq \Hom_\Cat(\varinjlim X(p), Y)\\
            &\simeq \Hom_\Cat(L_P(X), Y),
        \end{align*}
        where $\Hom_{\PreSheaves}\left(\Hom(\anon, p), \Hom_\Cat(X(\anon), Y)\right) \simeq \Hom_\Cat(X(p), Y)$ follows from the Yoneda lemma.
    \end{proof}

    By~\cref{prop:adjunction}, for any presheaf $P$, the functors $L_P, R_P$ induce a comonad $L_P\circ R_P$ on $\Cat$, which we denote $T_P$. Mostly, we will use the presheaf
    \begin{equation*}
        I = \coprod_{p \in \Obj(\PCat)} \Hom(\anon, p).
    \end{equation*}
    \begin{remark}
        In the case of the presheaf $I$, the comonad $T_I$ is equivalent to a construction derived in the context of {\em directed containers} in~\cite{Ahman_2016}.
    \end{remark}
   
    \begin{proposition}\label{prop:morphisms}
        Let $X, Y \in \Obj(\Cat)$. Then         
        \begin{equation*}
            \Hom_\Cat(T_I(X), Y) \simeq
            \prod_{p\in \Obj(\PCat)}
            \Hom_\Cat\left(\prod_{q \in \Obj(\PCat)} X^{\Hom(p, q)}, Y\right).
        \end{equation*}
    \end{proposition}
    \begin{proof}
        By~\cref{prop:adjunction}.
    \end{proof}

    \subsection{Recurrent neural networks}

    Let us consider the category $\mathbf N$, with one object whose endomorphisms are given by the monoid $\mathbb N$. Then the above construction recovers recurrent neural networks.

    %% Old simpler version

    Let $\JCat$ be a small category, and let $J = \Obj(\JCat)$ be its underlying set of objects. We can consider $J$ as a category, whose only morphisms are identities. Let $p \colon J \rightarrow \JCat$ be the inclusion of $J$ in $\JCat$. We are interested in extending functors $J \rightarrow \Cat$ to functors $\JCat \rightarrow \Cat$, using a standard construction, {\em right Kan extensions}~\cite[Chapt.~X]{mac2013categories}. For the inclusion $J \rightarrow \JCat$, the explicit formula given in~\cite[Sect.~3.7]{borceux1994handbook}, which requires $\Cat$ to be complete, simplifies in this scenario. In particular, we only require $\Cat$ to have small products. 

\begin{proposition}\label{prop:adjunction}
    The functor $p \colon J \rightarrow \JCat$ induces an adjunction
    \begin{align*}
        p^*\colon \Cat^\JCat & \rightarrow \Cat^J
                             &
                             & \dashv
                             &
        p_*\colon \Cat^J     & \rightarrow \Cat^\JCat                         \\
        F                    & \mapsto F\circ p
                             &
                             &
                             &
        G                    & \mapsto \prod_{j \in J} G(j)^{\Hom(\anon, j)}.
    \end{align*}
    That is to say, given functors $F\colon \JCat \rightarrow \Cat$ and $G \colon J \rightarrow \Cat$, there is a natural isomorphism
    \begin{equation*}
        \Hom(p^*F, G) \simeq \Hom(F, p_*G).
    \end{equation*}
\end{proposition}

\begin{proof}
    Thanks to Yoneda's lemma, for all $j \in J$,
    \begin{equation*}
        \Hom(F(j), G(j)) \simeq \Hom(\Hom(\anon, j), \Hom(F(\anon), G(j))).
    \end{equation*}
    As a consequence,
    \begin{align*}
        \Hom(p^* F, G)
         & \simeq \prod_{j\in J} \Hom(F(j), G(j))                            \\
         & \simeq \prod_{j \in J} \Hom(\Hom(\anon, j), \Hom(F(\anon), G(j))) \\
         & \simeq \prod_{j \in J} \Hom(F, G(j)^{\Hom(\anon, j)})             \\
         & \simeq \Hom(F, p_* G),
    \end{align*}
    where all isomorphisms are natural.
\end{proof}

By~\cref{prop:adjunction}, we can define a comonad $T = p^*p_* \colon \Cat^J \rightarrow \Cat^J$. The Kleisli category associated to $T$, denoted $\Cat^J_T$, can be described explicitly. Given $X, Y\in \Obj(\Cat^J)$, we have
\begin{equation*}
    \Hom(T X, Y) \simeq
    \prod_{i \in J} \Hom\left(\prod_{j \in J} X(j)^{\Hom(i, j)}, Y(i)\right).
\end{equation*}

\begin{remark}
    The comonad $T$ is remindful of constructions derived in the context of {\em directed containers} in~\cite{Ahman_2016,uustalu_comonadic_2008}.
\end{remark}

\begin{definition}
    Given a subset of morphisms $M \subseteq \Morph(\JCat)$, we can consider a functor $T_M$, given by
    \begin{align*}
        T_M \colon \Cat^J & \rightarrow \Cat^J                                    \\
        X                 & \mapsto \prod_{j \in J} X(j)^{\Hom(\anon, j) \cap M}.
    \end{align*}
    We denote $\mu$ the obvious natural transformations $T \rightarrow T_M$.
    We say that a map $f\colon TX \rightarrow Y$ is $M$-local if it factors via
    \begin{equation*}
        \mu_X \colon TX \rightarrow T_M X.
    \end{equation*}
\end{definition}
  
\bibliographystyle{abbrv}
\bibliography{References}

\end{document}