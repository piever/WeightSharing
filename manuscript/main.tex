\documentclass[12pt]{article}

\usepackage{amsmath, amsthm, amssymb}
\usepackage{amssymb}
\usepackage{bbm}

\usepackage[hidelinks]{hyperref}
\usepackage{cleveref}
\usepackage{url}

%%% For figures
\usepackage{graphicx}
\DeclareGraphicsExtensions{.pdf,.png,.jpg}

%%% For tables
\usepackage[table]{xcolor}

%specifying table and figure packages
\usepackage{caption}

\usepackage[T1]{fontenc}
\usepackage[all]{xy}
\usepackage[inline]{enumitem}

\usepackage{tikz-cd}

\newtheorem{theorem}{Theorem}
\newtheorem{corollary}{Corollary}
\newtheorem{remark}{Remark}
\newtheorem{example}{Example}
\newtheorem{definition}{Definition}
\newtheorem{proposition}{Proposition}
\newtheorem{lemma}{Lemma}

\newcommand{\mattia}[1]{\textcolor{cyan}{#1}}
\newcommand{\pietro}[1]{\textcolor{teal}{#1}}
\newcommand{\N}{{\mathbb{N}}}
\newcommand{\Ban}{{\mathbf{Ban}}}
\newcommand{\Top}{{\mathbf{Top}}}
\newcommand{\pt}{{\textnormal{pt}}}
\newcommand{\Hom}{{\textnormal{Hom}}}
\newcommand{\End}{{\textnormal{End}}}
\newcommand{\Fun}{{\textnormal{Fun}}}
\newcommand{\Aut}{{\textnormal{Aut}}}
\newcommand{\Obj}{{\textnormal{Obj}}}
\newcommand{\id}{{\textnormal{id}}}
\newcommand{\Morph}{{\textnormal{Morph}}}
\newcommand{\Set}{{\mathbf{Set}}}
\newcommand{\Cat}{{\mathbf{C}}}
\newcommand{\DCat}{{\mathbf{D}}}
\newcommand{\FCoalg}{{F\textnormal{-coalg}}}
\newcommand{\stCoalg}{{s_!t^*\textnormal{-coalg}}}
\newcommand{\range}[2]{{\{{#1}, \dots,{#2}\}}}
\newcommand{\anon}{{\,\mbox{-}\,}}

\crefname{diagram}{diag.}{diags.}
\Crefname{diagram}{Diagram}{Diagrams}
\creflabelformat{diagram}{#2(#1)#3}

\title{Weight sharing via comonads}
\author{
    Pietro Vertechi \and Mattia G. Bergomi
}
\date{}
\begin{document}
\maketitle
\begin{abstract}
\end{abstract}

\section{Introduction}

\pietro{Forse non va piu bene.}
One of the key reasons for the success of neural networks is the notion of {\em weight sharing}. In its simplest form, one can consider, among maps $[M, X] \rightarrow [M, X]$, those that originate from maps $[M, X] \rightarrow X$. Indeed, given
\begin{equation*}
    {\cal F} \colon [M, X] \rightarrow X,
\end{equation*}
one can consider the map
\begin{align*}
    \widehat{\cal F} \colon [M, X]       & \rightarrow [M, X]              \\
    \left(\widehat{\cal F}\phi\right)(m) & = {\cal F}(a \mapsto \phi(ma)).
\end{align*}

This simple notion of weight sharing does not seem sufficient to cover complex neural networks, made up of several nodes, each with their own symmetries, which interact with each other in non-trivial ways.

We aim to show that the simple notion of weight sharing defined above is sufficient to recover, among other examples, classical recurrent and convolutional neural networks. However, to do so, we will need to generalize not the weight sharing strategy, but rather the underlying category on which such strategy takes place.

\section{Categorical preliminaries}

\begin{definition}\label{def:nilpotent_functor}
    A functor $F$ is {\em nilpotent} if there exists $n \in \N$ such that the composed functor $F^n$ maps all objects to the terminal object.
\end{definition}

Graph-like structures are a source of examples of nilpotent functors.

\begin{proposition}\label{prop:graph_architecture}
    Let $(E, V)$ be a directed acyclic graph. Let $\DCat$ be a category with finite products. The functor
    \begin{align*}
        F \colon \DCat^V & \rightarrow \DCat^V                  \\
        (F X)(v)         & = \prod_{u \, | \, (u, v)\in E} X(u)
    \end{align*}
    is nilpotent.
\end{proposition}

\begin{proof}
    Let $n$ be such that there are no paths of length $n$ in $(V, E)$. Then $F^n X$ is the terminal object.
\end{proof}

\Cref{prop:graph_architecture} has a continuous analogue in locally Cartesian closed categories, which will be our focus.

\begin{definition}\label{def:categorical_architecture}
    Let $\DCat$ be a locally Cartesian closed category. An internal quiver
    \begin{equation*}
        s, t\colon E \rightrightarrows V
    \end{equation*}
    {\em has bounded depth} if there exists $n \in \N$ such that the iterated pullback
    \begin{equation}\label{eq:nilpotent_left_adjoint}
        \underbrace{E \times_V E \times_V \dots \times_V E}_{n \text{ factors}}
    \end{equation}
    is initial in $\DCat/V$.
\end{definition}

\begin{proposition}\label{prop:categorical_architecture}
    Let $\DCat$ be a locally Cartesian closed category. Let $s, t\colon E \rightrightarrows V$ be an quiver of bounded depth. Then \pietro{TODO: definisci * e ! (base change functors).}
    \begin{equation*}
        s_!t^*\colon \DCat/V \rightarrow \DCat/V
    \end{equation*}
    is a nilpotent endofunctor.
\end{proposition}

\begin{proof}
    The functor $s_!t^*$ is the right adjoint of the functor  $t_*s^*\colon \DCat/V \rightarrow \DCat/V$. By~\cref{eq:nilpotent_left_adjoint}, $(t_*s^*)^n$ maps all objects to the initial object, so its adjoint $(s_!t^*)^n$ maps all objects to the terminal object.
\end{proof}

\subsection{The category of coalgebras}

Let $\Cat$ be a Cartesian closed category with finite products, and let $F\colon \Cat \rightarrow \Cat$ be a finite product-preserving, nilpotent endofunctor. We can consider $\FCoalg$, the category of coalgebras over $F$ \pietro{add ref}. Objects are simply maps
\begin{equation*}
    X \rightarrow F X,
\end{equation*}
where $X \in \Obj(\Cat)$. Morphisms are given by commutative diagrams
\begin{equation*}
    \begin{tikzcd}
        X \arrow[rightarrow]{d} \arrow[rightarrow]{r}{\phi}
        & Y \arrow[rightarrow]{d} \\
        FX \arrow[rightarrow]{r}{F \phi}
        & FY
    \end{tikzcd}
\end{equation*}
As $F$ is nilpotent and preserves finite products, it is straightforward to verify that the forgetful functor
\begin{equation*}
    U \colon \FCoalg \rightarrow \Cat
\end{equation*}
has a right adjoint $\Phi$, called the cofree functor, given by
\begin{equation*}
    \Phi(X) = X \times F X \times \dots \times F^{n-1} X \rightarrow F X \times \dots \times F^{n-1} X.
\end{equation*}

% This adjunction is comonadic \pietro{explain! cite Beck's theorem!}, and $\FCoalg$ is the Eilenberg-Moore category $\Cat^T$, where $T = U\Phi$. As $T$ is left exact (product of left exact), $\FCoalg \simeq \Cat^T$ is Cartesian closed and it has finite limits, which are preserved by both $U$ and $\Phi$~\cite[Sect.~5.50]{wyler1991lecture}.

% \begin{proposition}\label{prop:cofree_exponentiating}
%     Let $F\colon \Cat \rightarrow \Cat$ be a left exact, nilpotent endofunctor. Then the exponential of cofree coalgebras is again cofree.
%     More explicitly, let $\Phi\colon \Cat \rightarrow \FCoalg$ denote the cofree functor. Then, there exists a natural isomorphism
%     \begin{equation*}
%         [B, \Phi X] \simeq \Phi[UB, X].
%     \end{equation*}
% \end{proposition}

% \begin{proof}
%     By a straightforward computation:
%     \begin{align*}
%         \Hom_{\FCoalg}(A, [B, \Phi X])
%          & \simeq \Hom_{\FCoalg}(A\times B, \Phi X)     \\
%          & \simeq \Hom_{\Cat}(U(A \times B), X)  \\
%          & \simeq \Hom_{\Cat}(UA \times UB, X)   \\
%          & \simeq \Hom_{\Cat}(UA, [UB, X])       \\
%          & \simeq \Hom_{\FCoalg}(A, \Phi[UB, X]).
%     \end{align*}
%     By Yoneda's lemma, this induces a natural isomorphism
%     \begin{equation*}
%         [B, \Phi X] \simeq \Phi[UB, X].
%     \end{equation*}
% \end{proof}

\section{Machines}

\begin{definition}\label{def:machine}
    Let $\Cat$ be a category, equipped with a terminal object $\pt$. Let $X \in \Obj(\Cat)$. A map
    \begin{equation*}
        \varrho \colon X \rightarrow X
    \end{equation*}
    is a {\em machine} if the induced transformation
    \begin{equation*}
        \Hom_\Cat(\pt, X) \rightarrow \Hom_\Cat(\pt, X)
    \end{equation*}
    has a unique fixed point $S_\varrho$, which we call the {\em stable state} of $\varrho$.
\end{definition}

\pietro{Fare qualche esempio, tipo contrazioni.}

\begin{definition}\label{def:parametric_machine}
    Let $\Cat$ be a category with finite products. A {\em parametric machine} is simply a machine on the Kleisli category for the comonad $P \times \anon$, for some $P \in \Obj(\Cat)$. More explicitly, a map
    \begin{equation*}
        \varrho \colon P \times X \rightarrow X
    \end{equation*}
    is a {\em machine} if there exists a unique map $S_\varrho$ such that the following diagram commutes
    \begin{equation}
        \begin{tikzcd}[column sep=large]\label[diagram]{diag:machine_condition}
            P \arrow[rightarrow]{r}{\left(\id, S_\varrho\right)}
            \arrow[rightarrow,swap]{dr}{S_\varrho}
            & P \times X \arrow[rightarrow]{d}{\varrho} \\
            & X
        \end{tikzcd}
    \end{equation}
\end{definition}

\Cref{diag:machine_condition} is analogous to the machine condition in~\cite{2020arXiv200702777V} in the case when $X$ is an Abelian group and $\varrho\colon P \times X \rightarrow X$ is the sum of a map $P \rightarrow X$ and a map $X \rightarrow X$. This approach differs from the one taken in~\cite{2020arXiv200702777V}, in that here the object $P$ represents both input space and parameters. In what follows, we will show how to construct examples of {\em machines} using {\em nilpotent functors}.

% \begin{definition}\label{def:nilpotent_architecture}
%     A {\em nilpotent architecture} is a left exact, nilpotent endofunctor $F$ on a finitely complete, Cartesian closed category $\Cat$.
% \end{definition}

\begin{theorem}\label{thm:nilpotent_architecture}
    Let $F\colon \Cat \rightarrow \Cat$ be a left exact, nilpotent endofunctor. Let $\Phi(X)$ be a cofree coalgebra in $\FCoalg$. Then a map $F X \rightarrow X$ induces a machine $\Phi X \rightarrow \Phi X$.
\end{theorem}

\begin{proof}
    \pietro{TODO, da fare anche con parametric machines e $P \times F X \rightarrow X$, dove $P$ è una $F$ coalgebra con un'azione di $M$.}
\end{proof}

\begin{definition}\label{def:architecture}
    An {\em architecture} is a quiver of bounded depth $s,t \colon E \rightrightarrows V$ in a locally Cartesian closed category $\DCat$, together with a monoid object $M$ in $\stCoalg$.
\end{definition}

This is sufficient to give notions of composability, weight sharing, and locality.

\begin{theorem}\label{thm:weight_sharing}
    A nilpotent architecture $M, s, t$ induces a comonad on $\DCat/V$, given by
    \begin{equation*}
        X \mapsto U\Phi[UM, X].
    \end{equation*}
\end{theorem}

\begin{proof}
    \pietro{Free forget composed with $[M, \anon]$ in $\stCoalg$.}
\end{proof}

\pietro{Forse si possono riunificare le due nozioni!}

\section{Recurrent and convolutional neural networks}

Monoids can be considered as categories with one object. Let us consider the category corresponding to $\N$. Then the above construction recovers recurrent neural networks. Analogously, considering $\N \times \N$ we obtain convolutional neural networks \pietro{discuss shape of filter?}

\section{Novel architectures}

\bibliographystyle{abbrv}
\bibliography{References}

\end{document}
